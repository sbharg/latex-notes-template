%%%%%%%%%%%%%%%%%%%%%%%%%%%%%%%%%%%%%%%%%%%%%%%%%%%%%%%%%%%%%%%%%%%%%%%%%%%%%%%%%%%%%%%%%%%%%%%%%%%%
% Custom commands -- Some math symbols and math shortcuts
%%%%%%%%%%%%%%%%%%%%%%%%%%%%%%%%%%%%%%%%%%%%%%%%%%%%%%%%%%%%%%%%%%%%%%%%%%%%%%%%%%%%%%%%%%%%%%%%%%%%

% --Shortcuts for left/right-----
\newcommand{\lrp}[1]{\left( #1 \right)}
\newcommand{\lrb}[1]{\left[ #1 \right]}
\newcommand{\lrc}[1]{\left\{ #1 \right\}}
\newcommand{\lrv}[1]{\left\langle #1 \right\rangle}
\newcommand{\abs}[1]{\left\lvert #1 \right\rvert}
\newcommand{\norm}[1]{\left\lVert #1 \right\rVert}
\newcommand{\ceil}[1]{{\lceil {#1} \rceil}}
\newcommand{\floor}[1]{{\lfloor {#1} \rfloor}}

% --Commonly-used sets-----------
\newcommand{\N}{\ensuremath{\mathbb{N}}}
\newcommand{\Z}{\ensuremath{\mathbb{Z}}}
\newcommand{\R}{\ensuremath{\mathbb{R}}}
\newcommand{\Q}{\ensuremath{\mathbb{Q}}}
\newcommand{\C}{\ensuremath{\mathbb{C}}}
\newcommand{\powerset}{\ensuremath{\mathcal{P}}}
\DeclareMathOperator*{\argmin}{arg\,min}
\DeclareMathOperator*{\argmax}{arg\,max}

% --Probability------------------
\DeclareMathOperator*{\E}{\mathbb{E}}
\newcommand{\EX}[1]{\ensuremath{\E \lrb{#1}}}
\newcommand{\PR}[1]{\ensuremath{\Pr \lrb{#1}}}

% --Complexity-------------------
\newcommand{\NP}{\textsf{NP}}
\newcommand{\NPH}{\textsf{NP-Hard}}
\newcommand{\NPC}{\textsf{NP-Complete}}
\newcommand{\ksum}{\textsf{$k$-Sum}}
\newcommand{\eqcarpart}{\textsf{Equal Cardinality Partition}}
\newcommand{\knapsack}{\textsf{Knapsack}}
\newcommand{\subsetsum}{\textsf{Subset Sum}}

\newcommand{\bigO}[2][]{\ensuremath{O_{#1}(#2)}}
\newcommand{\bigOlog}[2][]{\ensuremath{\tilde{O_{#1}}(#2)}}
\newcommand{\littleO}[2][]{\ensuremath{o_{#1}(#2)}}

\newcommand{\bigOmega}[2][]{\ensuremath{\Omega_{#1}(#2)}}
\newcommand{\littleOmega}[2][]{\ensuremath{\omega_{#1}(#2)}}

\newcommand{\bigTheta}[2][]{\ensuremath{\Theta_{#1}(#2)}}
\newcommand{\polyreduce}{\ensuremath{\leq_p}}

% --Custom-----------------------
\newcommand{\f}[2][f]{\ensuremath{#1 \! \lrp{#2}}}
\newcommand{\func}[2]{\f[#1]{#2}}

% --Graphs-----------------------
\newcommand{\degree}[2][]{\f[\mathrm{deg}_{#1}]{#2}}
\newcommand{\cut}[2][]{\f[\delta_{#1}]{#2}}

% --DiscreteOpt------------------
\newcommand{\OPT}{\ensuremath{\mathrm{OPT}}}
\newcommand{\poly}[1]{\f[\mathrm{poly}]{#1}}
\newcommand{\polylog}[1]{\f[\mathrm{polylog}]{#1}}

% --More sets--------------------
\newcommand{\nonneg}[1]{\lrb{#1}^+}
\def\sse{\subseteq}
\def\sm{\setminus}
\def\xbar{\bar{x}}
\def\calZ{\mathcal{Z}}
\def\calH{\mathcal{H}}
\def\calV{\mathcal{V}}
\def\calE{\mathcal{E}}
\def\calQ{\mathcal{Q}}
\def\calM{\mathcal{M}}
\def\calN{\mathcal{N}}
\def\calG{\mathcal{G}}
\def\calR{\mathcal{R}}
\def\calJ{\mathcal{J}}
\def\calL{\mathcal{L}}
\def\calP{\mathcal{P}}
\def\calB{\mathcal{B}}
\def\calA{\mathcal{A}}
\def\calC{\mathcal{C}}
\def\calX{\mathcal{X}}
\def\calY{\mathcal{Y}}
\def\calS{\mathcal{S}}
\def\calT{\mathcal{T}}
\def\calU{\mathcal{U}}
\def\calI{\mathcal{I}}
\def\calF{\mathcal{F}}
 
% --Algorithms-------------------
\newcommand{\AlgIn}{\Statex \textbf{Input:} }
\newcommand{\AlgOut}{\Statex \textbf{Output:} }
\newcommand{\Algreq}{\Statex \textbf{Require:} }

% --Misc-------------------------
% Usage: \tempnote{author}{note}
\newcommand{\tempnote}[2]{$\ll${\bf #1: }{\it #2}$\gg${\marginpar{\tiny{{\bf#1}}}}}
% Usage: \newedit[Optional Tag on Margin]{Comment}
\newcommand{\newedit}[2][]{\textcolor{OrangeRed}{#2}{\marginpar{\tiny{{\bf \textcolor{OrangeRed}{#1}}}}}}
\newcommand{\TODO}[1]{\textcolor{OrangeRed}{{\bf TODO:} #1}}
\newcommand{\NOTE}[1]{\textcolor{OrangeRed}{{\bf NOTE:}} #1}
\newcommand{\cc}{\textcolor{red}{\bf [citation(s)]}}
\newcommand{\email}[1]{\textsf{#1}}
\newcommand{\customcite}[1]{\par \noindent \cite{#1} \citeauthor{#1}, \citetitle{#1}, \citeyear{#1}}
\newcommand{\defequals}{\overset{\mathrm{def}}{=}}
\newcommand{\crefdefpart}[2]{%
    \hyperref[#2]{\namecref{#1}~\labelcref*{#1}.\ref*{#2}}%
}

\makeatletter
\newcommand{\leqnomode}{\tagsleft@true}
\newcommand{\reqnomode}{\tagsleft@false}
\makeatother

\makeatletter
\newcommand{\specificthanks}[1]{\@fnsymbol{#1}}% Inserts a specific \thanks symbol
\makeatother

%\end{comment}